%% Generelt
\newcommand{\g}{\cdot} % Prikprodukt, gangetegn
\newcommand{\kraft}[2]{#1_{\text{#2}}} % Pæn subscript
\newcommand{\e}{\mathcal{E}} % Skrevet E
\newcommand{\matlab}{\textsc{matlab} } %Pænt MatLab
\newcommand{\nline}{\nonumber \\ &}
\newcommand{\defeq}{\stackrel{\mathrm{def}}{=}}
%\newcommand{\defeq}{\coloneqq}

%% Matematik

\newcommand{\abs}[1]{\left| #1 \right|} % Numerisk værdi
\let\underdot=\d % omdøb indbygget kommando \d{} til \underdot{}
\renewcommand{\d}[2]{\frac{d #1}{d #2}} % afledt
\newcommand{\dd}[2]{\frac{d^2 #1}{d #2^2}} % dbl.afledt
\newcommand{\pd}[2]{\frac{\partial #1}{\partial #2}} %partiel afl.
\newcommand{\pdd}[2]{\frac{\partial^2 #1}{\partial #2^2}} %db.p.afl.
\newcommand{\R}{\mathbb{R}} % Reelle tal
\newcommand\ZERO{\makebox(0,0){\text{\huge0}}} % Stort 0 til matricer
\newcommand{\inner}[2]{\langle #1, #2 \rangle}
\DeclareMathOperator{\sgn}{sgn}
\newcommand{\fd}[2]{\frac{\delta #1}{\delta #2}} %functional deriv.

% Vektorer

\newcommand{\xyz}[3]{\begin{bmatrix} #1 \\ #2 \\ #3 \end{bmatrix}} %3D-vektor
\newcommand{\xy}[2]{\begin{bmatrix} #1 \\ #2 \end{bmatrix}} %2D-vektor
\let\vaccent=\v % Omdøb \v{} til \vaccent{}
\newcommand{\vect}[1]{\mathbf{#1}}
\renewcommand{\v}[1]{\ensuremath{\mathbf{#1}}} % Vektor med fed
\newcommand{\gv}[1]{\ensuremath{\mbox{\boldmath$ #1 $}}} % Vektor med græske bogstaver
\newcommand{\hatvec}[1]{\hat{\mathbf{#1}}} % Hatvektor
\newcommand{\ihat}{\boldsymbol{\hat{\textbf{\i}}}} % Enhedsvektor i
\newcommand{\jhat}{\boldsymbol{\hat{\textbf{\j}}}} % .. j
\newcommand{\khat}{\mathbf{\hat{k}}}  % .. k
\newcommand{\xhat}{\mathbf{\hat{x}}} % Enhedsvektor x
\newcommand{\yhat}{\mathbf{\hat{y}}} % .. y
\newcommand{\zhat}{\mathbf{\hat{z}}} % .. z
\newcommand{\grad}[1]{\gv{\nabla} #1} % Gradient
\let\divsymb=\div % Omdøb \div til \divsymb
\renewcommand{\div}[1]{\gv{\nabla} \cdot \v{#1}} % Divergens
\newcommand{\curl}[1]{\gv{\nabla} \times \v{#1}} % Curl
% Vil man tage div eller curl af græske bogstaver,
% skal man lade argumentetet være fx \gv{\mu} for µ-vektor


%PhD
\newcommand{\el}{\ensuremath{\text{el}}}
\newcommand{\qp}{\ensuremath{\text{qp}}}
\newcommand{\BCS}{\ensuremath{\text{BCS}}}
\newcommand{\tq}{{\delta{q}}}
\newcommand{\tk}{{\delta{k}}}
\newcommand{\gc}{\text{gc}}
% Thomas stuff
\DeclareMathOperator{\diag}{diag}
\DeclareMathOperator{\Tr}{Tr}
\DeclareMathOperator{\Res}{Res} \DeclareMathOperator{\hc}{h.c.}
\DeclareMathOperator{\cc}{c.c.}
\renewcommand{\Re}{\operatorname{Re}} % can't use declaremathoperator
\renewcommand{\Im}{\operatorname{Im}} % because they're already defined
\newcommand{\tvk}{{\delta{\v{k}}}}
\newcommand{\NSN}{\text{NSN}}
\newcommand{\DeltaP}{\Delta_{\text{p}}}
\newcommand{\tDeltaP}{\tilde\Delta_{\text{p}}}
\newcommand{\vkappa}{{\boldsymbol{\kappa}}}

% Color
\def\bl{\textcolor{blue}}

% Acronyms
\setglossarysection{section=section} % For symbols
\setacronymstyle{long-short}
\newacronym{QH}{QH}{quantum Hall}
\newacronym{SC}{SC}{superconductor}
\newacronym{2DEG}{2DEG}{two-dimensional electron gas}
\newacronym{SOC}{SOC}{spin-orbit coupling}

% Symbols, chap 1
\newglossaryentry{e}{
	name=\ensuremath{e},
	description={electron charge},
	type=symbols
	}
\newglossaryentry{B0}{
	name=\ensuremath{B_0},
	description={applied magnetic field},
	type=symbols
	}